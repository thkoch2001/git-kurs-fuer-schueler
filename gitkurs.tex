\documentclass[11pt,a4paper,twoside]{scrartcl}

\usepackage[utf8]{inputenc}
\usepackage{url}
\usepackage[paper=a4paper,inner=20mm,outer=20mm,top=20mm,bottom=20mm]{geometry}
\usepackage{courier}
\usepackage{listings}

\lstset{
  basicstyle=\ttfamily
}

\begin{document}

\begin{center}
  {\LARGE Git Kurs \par}
\end{center}

\section{Einleitung}

In der Softwareentwicklung arbeiten meistens viele Entwickler zusammen an
einem Projekt. Eines der größten freien Projekte ist der Kernel des freien
Betriebssystems Linux mit über 1000 Entwicklern aus mehr als 30 Ländern. Alle
diese Menschen müssen ihre Arbeit an den gleichen Dateien koordinieren und
dabei noch die Übersicht über verschiedene Entwicklungsstände (Versionen)
behalten. Zu diesem Zweck hat der Gründer des Linux Projekts, Linus Torvalds,
das Versionskontrollsystem Git entwickelt.

Git lässt sich aber nicht nur zur kooperativen Softwareentwicklung nutzen
sondern überall wo mehrere Menschen gemeinsam Dateien bearbeiten. Auch kann
Git als eine Backuplösung verwendet werden so dass man immer zu einer älteren
Version einer Datei zurückspringen kann.

\emph{Suchen Sie im Internet nach weiteren Informationen über Git und Linux! Woher
  kommt der Name Git? Was ist das Maskottchen von Linux?}

\section{Ein neues Repository}

Git verwaltet Dateien in einem Verzeichnis und dessen
Unterverzeichnissen. Legen Sie daher bitte ein neues Verzeichnis an (Befehl
\lstinline{mkdir Verzeichnisname}), das danach zu einem sogenannten Repository gemacht
werden soll. Öffnen Sie eine Konsole und wechseln Sie in das soeben angelegte
Verzeichnis (Befehl \lstinline{cd Verzeichnisname}). Überprüfen Sie mit dem Befehl \lstinline{ls -a}
\footnote{Mit der Option -a zeigt der Befehl ls auch versteckte Dateien und
  Verzeichnisse an}, dass das Verzeichnis leer ist.

Ein neues Git repository im aktuellen Verzeichnis erzeugt der Befehl
\lstinline{git init}. Danach können Sie mit \lstinline{ls -a} sehen, dass ein
neues Verzeichnis ``.git'' angelegt wurde, in dem Git seine Hilfsdateien
ablegt. Der nächste Befehl, \lstinline{git status}, bestätigt, dass in dem
Repository noch nichts passiert ist:

\begin{lstlisting}
# On branch master
#
# Initial commit
#
nothing to commit (create/copy files and use "git add" to track)
\end{lstlisting}

\section{Dateien hinzufügen}

Die von Git verwalteten Versionen werden ``Commit'' und das Anlegen einer
neuen Version ``Committen'' genannt. Legen Sie mit dem Befehl
\lstinline{touch} gefolgt von einem Dateinamen eine leere Datei an. Mit dem
Befehl \lstinline{git add} gefolgt vom Dateinamen wird diese Datei zum
nächsten Commit hinzugefügt.

Erst mit dem Befehl \lstinline{git commit} wird tatsächlich eine neue Version
in Git angelegt. Jeder Commit muss auch eine kurze Erklärung erhalten, was in
dieser Version verändert wurde. Daher öffnet der Aufruf von git commit einen
Texteditor in dem man diese Erklärung eingeben kann, speichert und den Editor
schließt. Alternativ kann man mit der Option \lstinline{-m} direkt eine
Erklärung angeben, z.B.:

\begin{lstlisting}
git commit -m "3. Kapitel hinzugefuegt"
\end{lstlisting}

Git hat sich nun darüber beschwert, dass Sie sich noch nicht vorgestellt
haben. Wir ignorieren diese Beschwerde im Moment.

Committen Sie weitere Dateien in ihr Repository. Überprüfen Sie nach jedem
Befehl den Stand und die Historie Ihres Repositories mit den Befehlen
\lstinline{git status} und \lstinline{git log}\footnote{Die Anzeige eines
  längeren Logs wird mit der Taste q verlassen.}!

\section{Dateien bearbeiten}

Kopieren Sie eine LaTeX Datei (Endung ``.tex'') in das Repository. Machen Sie
Änderungen an der Datei und comitten Sie neue Versionen der Datei mit
\lstinline{git add} und \lstinline{git commit}. Der Befehl
\lstinline{git log -p} (Option \lstinline{-p} für ``patch'')
zeigt die Änderungen der einzelnen
Commits an.

Mit dem Befehl \lstinline{git rm} können Sie Dateien für den nächsten Commit
löschen. Löschen Sie die mit \lstinline{touch} angelegten leeren Dateien mit
\lstinline{git rm} und committen Sie die neue Version!

\section{Graphische Tools}

Die Konsole ist für den geübten Nutzer das schnellste und mächtigste
Werkzeug. Trotzdem bevorzugen Anfänger oft grafische Werkzeuge. Zum Betrachten
und Bearbeiten eines Git Repositories gibt es unter Linux die Programme gitk,
qgit, giggle, gitg. Probieren Sie aus, ob diese Programme bei Ihnen
installiert sind! Wechseln Sie dazu in der Konsole (mit \lstinline{cd}) in ein
Git Repository und geben Sie den Programmnamen als Kommando ein.

Zwei weitere grafische Tools, git gui und git-cola sollten auch installiert
sein. Sie sind eher zur Bearbeitung und zum Committen gedacht.

\section{Kontakt mit anderen}

Am einfachsten ist es, für die Zusammenarbeit einen von zahlreichen
\footnote{\url{https://git.wiki.kernel.org/index.php/GitHosting}} Git Hosting
Dienst zu benutzen\footnote{Wenn man einen eigenen Server betreibt kann man
auch sehr einfach einen eigenen Git Hosting Dienst einrichten.}. Der
bekannteste und am einfachsten zu benutzende Dienst ist \url{github.com}. Dort
findet man auch eine umfangreiche Hilfe für die nächsten Schritte:

\begin{enumerate}
  \item \url{https://help.github.com/articles/set-up-git}
  \item \url{https://help.github.com/articles/create-a-repo}
  \item \url{https://help.github.com/articles/fork-a-repo}
\end{enumerate}

\emph{Welche Daten bekommt Github von seinen Nutzern? Wie einfach kann man eine
Sicherheitskopie aller seiner wichtigen Daten von Github auf den eigenen
Rechner bekommen? Vergleiche die Situation mit Facebook!}

\end{document}
